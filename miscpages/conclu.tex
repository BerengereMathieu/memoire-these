%------------------------------%
%------------------------------%
 \chapter{Conclusion}
%------------------------------%
%------------------------------%
%\thispagestyle{plain}
%\addcontentsline{toc}{chapter}{Conclusion}
%\markboth{Conclusion}{Conclusion}

Dans ce mémoire, nous avons présenté une synthèse de nos travaux menés autour deux thématiques  différentes : la segmentation interactive et la sur-segmentation. Le fil \modif{conducteur} de ces recherches est la mise au point d'un outil de sélection et d'annotation permettant l'enrichissement d'un observatoire photographique du paysage. Ce dernier permet d'associer à une photographie une segmentation sémantique indiquant pour chaque pixel \modif{la catégorie d'objet à laquelle} il appartient.

 
\section{Segmentation interactive}

Ces deux dernières décennies\modif{,} de très nombreuses méthodes de segmentation interactive ont été proposées. Traditionnellement, ces méthodes sont classés selon leur mode d'interaction, avec des algorithmes axés sur la recherche des contours \cite{mille2015combination,mortensen1995intelligent} tandis que d'autres se basent sur celle des régions \cite{jian2016interactive,mcguinness2010comparative,santner2010interactive}. 

Dans ce \modif{mémoire,} nous avons montré que le choix du mode d'interaction influence le type de problème que la méthode pourra résoudre. Tandis que les méthodes par recherche des contours se limitent à la sélection d'\modif{objets correspondant à} une seule composante connexe, sans trou, le mode d'interaction propre aux méthodes par recherche des régions n'impose aucune limitation sur la topologie des objets ou leur nombre. 

Les trois catégories que nous \modif{avons proposé} dans le chapitre 2 \modif{--} méthodes de binarisation interactive par recherche des contours, méthodes de binarisation interactive par recherche des régions, méthodes de segmentation interactive \modif{multiclasse} \modif{--} permettent d'identifier le type de problème résolu par l'algorithme.

\subsection{Proposition d'une nouvelle méthode}
L'une des \modif{principales} contributions de nos travaux est la conception d'une nouvelle méthode de segmentation interactive \modif{multiclasse}, $S \alpha F$. Décrite en détail dans le chapitre 3, cette dernière repose sur l'extraction de primitives visuelles nommées superpixels, qui correspondent à de petites \modif{régions décrites} par leur couleur moyenne et la position de leur barycentre. 

L'utilisation d'un SVM, entraîné à partir des germes donnés par l'utilisateur, permet de tirer le meilleur parti de ce descripteur minimaliste. Nous montrons comment estimer à la fois la probabilité qu'un superpixel \modif{$\mathbf{s}_{i}$} appartienne à une classe $\lambda_{m}$ et celle que deux superpixels voisins,  \modif{$\mathbf{s}_{i1}$} et \modif{$\mathbf{s}_{i2}$} appartiennent respectivement aux classes $\lambda_{m1}$ et $\lambda_{m2}$. Ces deux types de probabilités nous permettent de calculer les facteurs d'une fonction que nous minimisons dans un graphe de \modif{facteurs}, afin de trouver une segmentation \modif{adéquate} de l'image.

\subsection{Mise à disposition d'un nouvel ensemble de données de référence}
L'évaluation et la comparaison des différentes méthodes de segmentation interactive est une tâche difficile. Les codes sources sont rarement disponibles, les données de référence \modif{sont} utilisées selon des modalités très différentes et l'ergonomie des méthodes est prise en compte de manière sporadique. Un \modif{travail} reste à faire qui permettrait une analyse rigoureuse des avantages et inconvénients de chaque algorithme.

Notre contribution concerne la création et \modif{la} mise à disposition de trois ensembles de données comprenant des images de tailles différentes afin de tester le passage à l'échelle d'une méthode de segmentation interactive \modif{multiclasse}. Les \modif{images} du dernier ensemble, contenant $1800 \times 1201$ pixels, ont en outre l'avantage de se rapprocher \modif{davantage} des dimensions des photographies prises par les appareils \modif{pour le grand public} que les données de \modif{référence utilisées jusqu'à maintenant} \cite{mcguinness2010comparative,rother2004grabcut, santner2010interactive}.


\subsection{Évaluation de la méthode proposée}
\modif{La} comparaison d'une nouvelle méthode de segmentation interactive avec les méthodes existantes est une tâche délicate. En mettant en regard les résultats obtenus par $S \alpha F$ avec ceux de l'état de l'art, nous avons montré l'intérêt de l'algorithme que nous proposons. 

Nous nous sommes également intéressés à certaines de ses propriétés :
\begin{itemize}
\item intérêt du terme de régularisation ;
\item passage à l'échelle ;
\item ergonomie ;
\item caractéristiques des germes à donner. 
\end{itemize}

Ces tests \modif{ont permis de mettre en évidence} les capacités de $S \alpha F$, ses points forts et ses faiblesses.


\section{Sur-segmentation}

\subsection{Conception d'un nouveau protocole d'évaluation}
L'une des principales motivations des algorithmes de sur-segmentation est de créer des groupes de pixels \modif{homogènes}, les superpixels, moins nombreux que les pixels, donc plus \modif{rapides} à traiter que ces derniers. Cependant, les données de référence utilisées pour évaluer les méthodes de sur-segmentation sont des images de taille modeste : à peine quelques milliers de pixels. 

Afin d'y remédier, nous avons \modif{conçu} et mis à disposition un nouvel ensemble de données, $HSID$, \modif{composé} d'images de tailles diverses, avec toutefois une majorité de grandes images, contenant plusieurs millions de \modif{pixels}. Les particularités de ces données nous ont conduit à repenser les mesures traditionnellement \modif{utilisées} pour évaluer la \modif{qualité} d'une sur-segmentation. 

Enfin\modif{,} une analyse des méthodes utilisant les superpixels nous \modif{a} permis de formaliser les attentes propres à un algorithme de sur-segmentation, au travers de cinq propriétés (validité, adhérence aux contours, concision, simplicité, rapidité) et du concept d'adaptabilité. L'ensemble de ce travail \modif{a été} détaillé \modif{au} chapitre 4.

\subsection{Analyse des méthodes existantes}

Présentée dans le chapitre 4, l'analyse des résultats obtenus par les \modif{méthodes de l'état de l'art} appliquées \modif{aux} images \modif{de} $HSID$ \modif{révèlent} que seulement trois d'entre elles conservent leurs bonnes performances lorsqu'elles sont \modif{amenées} à traiter des images de tailles différentes ou des images de grande taille. 

Le choix d'une méthode parmi les trois obtenant de \modif{bons résultats} dépend du domaine d'application. En particulier nos test\modif{,} montrent que :
\begin{itemize}
\item l'algorithme ERS de Liu \textit{et al.} \cite{liu2011entropy} permet d'obtenir une excellente adhérence aux contours avec un nombre peu élevé de superpixels. Cependant\modif{, cette méthode} est lente et n'a aucune adaptabilité ;
\item l'algorithme FZ de Felzenszwalb \textit{et al.} \cite{felzenszwalb2004efficient} fournit un compromis intéressant : bénéficiant d'une bonne adaptabilité, il est capable d'atteindre une adhérence aux contours \modif{correcte} avec un nombre de superpixels raisonnable. Ses temps \modif{d'exécution} sont trois fois plus \modif{rapides} que ceux de l'algorithme ERS. 
\item l'algorithme SLIC d'Achanta \textit{et al.} \cite{achanta2012slic} est de loin la méthode la plus rapide, avec un temps d'exécution dix fois inférieur à celui d'ERS. Il requiert toutefois quelques superpixels supplémentaires pour obtenir la même adhérence aux contours que \modif{ses} deux \modif{concurrents}. 
\end{itemize}

\subsection{Proposition d'une nouvelle méthode}

L'étude des algorithmes \modif{de sur-segmentation existants} nous a conduit à proposer, dans le chapitre 6, une nouvelle méthode, $ASARI$. Cette dernière s'appuie sur un premier résultat obtenu grâce à l'algorithme SLIC, qui correspond à une sur-segmentation en de très nombreux superpixels. Certains de ces superpixels sont alors fusionnés afin d'aboutir à un résultat meilleur que celui de SLIC en ce qui concerne l'adhérence aux contours et le nombre de superpixels, tout en conservant des temps d'exécution aussi faibles que possible. 

Les deux principales particularités d'ASARI sont :
\begin{itemize}
\item le fait que les superpixels produits \modif{soient non} seulement homogènes au sens de la couleur mais également au sens de la texture ;
\item une bonne adaptabilité, qui permet d'obtenir de grands superpixels dans les zones uniformes et de nombreux petits superpixels dans celles riches en détails. 
\end{itemize}

Les résultats d'ASARI sont \modif{encourageants}, en particulier pour les images de grandes tailles comme celle de $HSID$ pour lesquelles ses performances sont nettement meilleures que celle du reste de l'état de l'art.


\section{Perspectives}


\subsection{Évaluation des méthodes de segmentation interactive}

Si l'évaluation que nous avons conduite permet de montrer l'intérêt de $S \alpha F$, une comparaison plus \modif{complète} des divers algorithmes de l'état de l'art reste à réaliser. Elle nécessite un travail conséquent pour implémenter les méthodes existantes (\modif{leur code} n'étant pas disponible) et la mise en place d'un protocole permettant de mieux prendre en compte la place de l'utilisateur. 

De manière similaire à ce qui a été réalisé par McGuinness \textit{et al.} \cite{mcguinness2010comparative}, il serait souhaitable que chaque méthode soit évaluée par différents utilisateurs. Les questions \modif{posées} à ces derniers devront permettre d'obtenir des données quantitatives plus fiables concernant les différents aspects que nous avons évoqués dans le chapitre 5, avec notamment :

\begin{itemize}
\item l'influence du temps d'exécution sur la pénibilité d'utilisation d'un outil de segmentation interactive ;
\item les caractéristiques des germes à donner à chaque méthode et la manière dont ils permettent d'obtenir plus ou moins rapidement un résultat \modif{satisfaisant}. 
\end{itemize}

Par ailleurs, une telle évaluation nécessitera de poursuivre notre effort pour la mise à disposition d'ensembles de données de référence contenant des images de \modif{tailles similaires} à celles produites actuellement par les appareils photographiques grand \modif{public}. Les ensembles que nous avons proposés doivent être étendus. Il serait également souhaitable de \modif{travailler} à la réalisation d'un ensemble contenant des données de très grandes tailles (plusieurs \modif{milliers} de pixels \modif{en largeur et en hauteur}).

\subsection{Intégration d'ASARI au sein de $S \alpha F$}

Comme évoqué \modif{à la} fin du chapitre 5, l'étape de $S \alpha F$ ouvrant le plus de \modif{perspectives} pour l'amélioration de ses performances est celle de la sur-segmentation. Dans le chapitre 6\modif{,} nous avons présenté un nouvel algorithme de sur-segmentation : ASARI. 

Actuellement\modif{,} ASARI est trop lent pour être intégré avec succès au sein d'une méthode comme $S \alpha F$. En particulier\modif{, ses} temps d'exécution compromettent une implémentation \modif{utilisable} sur smartphone. Un travail important doit donc être réalisé, afin \modif{d’accélérer ASARI}. La conclusion du chapitre 6 nous a donné l'occasion d'évoquer quelques pistes, telles que la parallélisation de \modif{certaines} portions de l'algorithme. 

Une fois ce travail effectué, l'intégration d'ASARI \modif{à la méthode} $S \alpha F$ reposera la question du descripteur, les superpixels \modif{produits} par cet algorithme ayant la propriété de ne pas être seulement \modif{homogènes} au sens de la couleur, mais également en termes de texture. La mise en place de tests \modif{supplémentaires} devra permettre de décider si l'utilisation d'un descripteur plus complexe, intégrant la texture en plus de \modif{la} couleur et de la localisation, influence la qualité des segmentations produites ou les germes à donner.

\subsection{Observatoires photographiques du paysage }
 
 La version de $S \alpha F$ que nous avons décrite au chapitre 3 a été implémentée avec succès \modif{dans} une application distribuée facilitant l'enrichissement d'un observatoire photographique du paysage \cite{puel2017une}. L'algorithme $S \alpha F$ permet d'\modif{associer} à chaque photographie une segmentation indiquant la classe sémantique des pixels et permettant d'identifier et de localiser avec précision les objets \modif{présents}. La faible complexité de $S \alpha F$ permet une utilisation fluide \modif{de cet outil}. 
 
 Nous souhaitons tester l'utilisation de cette application \modif{distribuée} dans un contexte de jeu-sérieux où la mise en place d'un observatoire et la contribution à \modif{son amélioration serviraient} de supports dans des cours sur l'aménagement du territoire.  Nous envisageons d’ajouter à l’application mobile une fonctionnalité d’envoi sur le réseau social Twitter, afin qu'un message accompagne chaque  nouvelle image ajoutée à l'observatoire. De la sorte, les utilisateurs qui suivent le mot-clé correspondant seront informés des changements et les participants à un jeu sérieux seront avisés de l’avancée de leurs partenaires ou \modif{de leurs} concurrents.  Nous allons mener des tests avec deux classes et leurs enseignants,  l'une en \modif{classe de terminale} technologique \modif{\og Sciences et technologies de l’agronomie et du vivant \fg}, l'autre en \modif{en classe de terminale} professionnelle \modif{\og Aménagements paysagers \fg.} 