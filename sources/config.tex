
%% packages utilises

%%---------------------

\usepackage{etex}

\usepackage[utf8]{inputenc}
\usepackage[T1]{fontenc}
\usepackage[french]{babel}
\addto\captionsfrench{\def\tablename{{\scshape Tableau}}}
\addto\captionsfrench{ \renewcommand\listfigurename{Liste des figures}}
\usepackage[french]{translator}
\usepackage{frbib}
\usepackage{amsmath}
\usepackage{amssymb}
\usepackage{these}

\usepackage[a4paper]{geometry}

\usepackage{ulem}
\normalem
\usepackage{rotating}
\usepackage{tabularx}
\usepackage{textcase}
\usepackage{graphicx}
\usepackage{epstopdf}		% .eps to .pdf
\usepackage{moreverb} %% pour le verbatim en boite
\usepackage{multirow} %% pour regrouper un texte sur plusieurs lignes dans une table
\usepackage{url} %% pour citer les url par \url
\urlstyle{sf}
\usepackage[all]{xy} %% pour la barre au dessus des symboles
\usepackage{shorttoc} %% pour plusieurs tables des matières par la commande \shorttableofcontents{Titre}{profondeur}.
\usepackage{textcomp} %% pour le symbol pour mille par \textperthousand.

\usepackage[right]{eurosym}
\usepackage{eufrak}
\usepackage{mathrsfs}
\usepackage{latexsym}
% 
\usepackage{setspace} %allow to change line spacing
\usepackage{fancyhdr} %header / footer

\usepackage{enumitem} %beautiful enumerations ... no beamer
\usepackage[singlelinecheck=false ]{caption} %
\usepackage{subcaption}
\usepackage{color} %use color
\usepackage{xcolor}
\usepackage{colortbl}
\usepackage{epsfig,amsfonts} %math mode
\usepackage{array,colortbl} %use array / tabular and apply color / style on them
\usepackage{pstricks} %draw with latex
\usepackage{tikz}
\usepackage{listings} %source code with avec \being{lstlistgin}

\usepackage{stmaryrd}
\usepackage{tocbibind}
%\usepackage[nottoc]{tocbibind}
\usepackage{pdfpages}
\usepackage{chngpage}
\usepackage{multicol}
 
\usepackage[Algorithme]{algorithm}
\usepackage{algpseudocode}

\usepackage[autolanguage]{numprint}

\usepackage{hyperref}
\usepackage[toc]{glossaries}

%\usepackage[toc,nonumberlist,translate=babel]{glossaries}

\usepackage[resetlabels]{multibib}
\newcites{publis}{Bibliographie personnelle}

\usepackage{bm}

%% choix des profondeurs de section pour la table des matières
%% 2= subsection, 3=subsubsection
\setcounter{secnumdepth}{2}  %% Avec un numero.
\setcounter{tocdepth}{2}     %% Visibles dans la table des matieres

\makeglossary

\def\underscore{\char`\_}

\usepackage[refpage]{nomencl}
\renewcommand{\nomname}{Liste des notations}
\renewcommand*{\pagedeclaration}[1]{\unskip\dotfill\hyperpage{#1}}
\makenomenclature

\usepackage{makeidx}
\makeindex


\definecolor{colKeys}{rgb}{0,0,1}
\definecolor{colIdentifier}{rgb}{0,0,0}
\definecolor{colComments}{rgb}{0,0.5,1}
\definecolor{colString}{rgb}{0.6,0.1,0.1}

\definecolor{c1}{RGB}{205,204,0}
\definecolor{c2}{RGB}{86,170,96}
\definecolor{c3}{RGB}{160,44,78}
\definecolor{c4}{RGB}{55,171,200}

\definecolor{gris}{gray}{0.75}

\DeclareMathOperator*{\argmax}{arg\,max}
\DeclareMathOperator*{\argmin}{arg\,min}
\DeclareMathOperator*{\pow}{pow}
\DeclareMathOperator*{\nei}{\mathcal{F}_{voi}}
\DeclareMathOperator*{\signature}{signature}
\DeclareMathOperator*{\acos}{acos}
\DeclareMathOperator*{\tr}{tr}

\usepackage{csquotes}
\lstset{%configuration de listings
	language=C,
	basicstyle=\ttfamily\small, %
	identifierstyle=\color{colIdentifier}, %
	keywordstyle=\color{colKeys}, %
	stringstyle=\color{colString}, %
	commentstyle=\color{colComments}, %
	columns=flexible, %
	tabsize=3, %
	%frame=trBL, %
	extendedchars=true, %
	%showspaces=false, %
	showstringspaces=false, %
	numbers=none, %
	breaklines=true, %
	breakautoindent=true, %
	captionpos=b,%
	xrightmargin=0cm, %
	xleftmargin=0cm,
	mathescape=true
}
\frenchspacing

\newgeometry{hmargin={0pt,0pt},vmargin={0pt,0pt}}
\savegeometry{include}

\newgeometry{vmargin={4.1cm,3.6cm},hmargin={3cm,2cm},twoside}
\setstretch{1.1}
\savegeometry{normalpreprint}

\newgeometry{vmargin={4.1cm,3.6cm},hmargin={2cm,3cm},twoside}
\setstretch{1.1}
\savegeometry{normalpreprintinverse}

\newgeometry{vmargin={4.1cm,3.6cm},hmargin={2.5cm,2.5cm}}
\setstretch{1.1}
\savegeometry{normaldigital}

\ifthenelse{\equal{\preprint}{YES}}
{
\loadgeometry{normalpreprint}
\savegeometry{normal}
\loadgeometry{normalpreprintinverse}
\savegeometry{normalinverse}
}
{
\loadgeometry{normaldigital}
\savegeometry{normal}
\savegeometry{normalinverse}
}


\setlist[itemize,1]{label=$\blacksquare$}
\setlist[itemize,2]{label=$\bullet$}

%% macro/racourcis por les symboles et commandes usuelles
\input{sources/macro.tex}

\title{\TITRE}

\author{\AUTp \AUTn}

\hypersetup{
	pdfauthor={{\AUTp} {\AUTn}},
	pdftitle={{\TITRE}},
	pdfsubject={\MENTION},
	pdfkeywords={\KEYWORDS},
	hidelinks
}

\usepackage{titlesec}
 \usepackage{anyfontsize}
 
 
\titleformat{\chapter}[display]
  {\bfseries}
  {\filleft \fontsize{50}{40}\selectfont \textbf{Chapitre}  \fontsize{100}{60}\selectfont \textbf{\textcolor{c4}{ \thechapter }}{ }}
  {1ex}
  {
  \vspace{1ex}\filright   \Huge}
  [\vspace{1ex} \textcolor{c4}{\titlerule}]
 
 %%%%%%%%%%%%%%%%%%%%%%
% Gestion des flottants (figures, tables, etc.)
%
% proportion minimale d'une page normale devant être occupée par du texte (défaut 0.2)
\renewcommand{\textfraction}{0.01} %0.05
% proportion maximale de la page pouvant être occupée par des flottants en haut de la page (défaut 0.7)
\renewcommand{\topfraction}{0.99} %0.95
% proportion maximale de la page pouvant être occupée par des flottants en bas de la page (défaut 0.3)
\renewcommand{\bottomfraction}{0.99} %0.95
% proportion minimale d'une page de flottants devant être occupée par des flottants (défaut 0.5) ; permet de limiter les espaces vides dans les pages de flottants
\renewcommand{\floatpagefraction}{0.55} %0.35
% nombre maximal de flottants permis sur une seule page (défaut 3)
\setcounter{totalnumber}{5}
%%%%%%%%%%%%%%%%%%%%%%


\newcommand{\modif}[1]{{#1}} % comment to cancel colorization
\newenvironment{emodif}{}  % comment to cancel colorization

